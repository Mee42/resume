\documentclass[10pt,letterpaper]{article}


\usepackage[letterpaper,margin=0.75in]{geometry}
\usepackage[utf8]{inputenc}
\usepackage{mdwlist}
\usepackage[T1]{fontenc}
\usepackage{textcomp}
\usepackage{tgpagella}
\usepackage{minted}
\usepackage{multicol}
\usepackage{hyperref}

\pagestyle{empty}
\setlength{\tabcolsep}{0em}

\title{}
\author{Carson Graham}
\date{}


\hypersetup{
  colorlinks   = true, %Colours links instead of ugly boxes
  urlcolor     = blue!60!black, %Colour for external hyperlinks
  linkcolor    = blue!50!black, %Colour of internal links
  citecolor    = red %Colour of citations
}

\renewcommand\labelitemi{---} % this is the list item hyphen


% make "C++" look pretty when used in text by touching up the plus signs
\newcommand{\CPP}
{C\nolinebreak[4]\hspace{-.05em}\raisebox{.35ex}{\footnotesize\bf ++}}

% write \code{} and have it formamtted as code. Only for inline text
\def\code#1{\texttt{#1}}

\begin{document}

    \maketitle

    \section*{Contact info}

        You can reach me through email at \href{mailto:carson42g@gmail.com}{carson42g@gmail.com}
        and my GitHub account \href{https://github.com/mee42/}{mee42}

    \section*{Skills}

    \begin{itemize}
        \item Extensive experience with Java and Kotlin, two JVM languages.
        \item Experience with low-level langauges C, \CPP, and x86\_64 Assembly.
        \item Experience with functional programming paradigms in the forms of Haskell and Kotlin
        \item Experience with standard GNU/Linux systems and programs, specifically Debian and Arch Linux.
        \item Minor experience in scripting languages such as Bash and Python.
        \item Experience with build tools such as Makefile, Makepkg, and Gradle.
        \item Leadership experience (see: Robotics, CSCS)
    \end{itemize}

    \hrule
    \vspace{-0.4em}

    \section*{Leadership}
    \subsection*{Oakton FRC Robotics Club}

    I am one of the two programming leads for the Oakton HS FRC team,
    a competitive robotics competition.
    During season, which is 6-10 weeks, we meet about 10-30 hours a week.
    I was the main lead and, along with our 2 existing members,
    helped our 3 new programmers learn both \CPP and Git.
    Under my guidance we adopted a mix of git-flow and trunk-based development,
    which is documented on the wiki \href{https://github.com/CougarProgramming623/InfiniteRecharge/wiki/Git}{here}.

    \subsection*{Oakton CSCS Club}

    I'm the president of the Oakton CSCS (Computer Science and Cyber Security) Club,
    one of the largest student-run clubs at my highschool. 
    We have a team of 6 officers and a club of about 50 offical members -- and 150 members on our public Discord server.
    We go to Cyber Patriot, as well as several other CTFs and Computer Science competitions, and we have plans to host our own hackathon.
    As the president, my job is to make sure everything gets done,
    which is mainly planning meetings and club projects.

    \vspace{1em}
    \hrule

    \section*{Formal Education}
    \begin{itemize}
        \item AP Computer Science A, \textit{Intro to Java and Compure Science}
        \item Advanced Computer Science AB, \textit{Data Structures and Algorithms}
        \item Oracle Assocate Certification, \textit{Foundations to Java SE 8}
    \end{itemize}

    \vspace{0.5em}
    \hrule

    \section*{Personal Projects}

    \subsection*{Vision Pipeline}
    \href{https://github.com/CougarProgramming623/vision-pipeline}{Github}. 
    
    In robotics, being able to estimate the robot's prose in relation to the target
    allows you to autonomusly drive the robot better then a human can from 50 feet away.
    In the 2020 season, the goal was to shoot a ball through a hole from about 20 feet away,
    and the hope was to use a camera and some code to aim at the goal and fire with accuracy.
    unfortunatly, this wasn't production-ready by the competition,
    but I still learned a lot about opencv and solvepnp-type problems. 

    \subsection*{Xenon}
    \href{https://github.com/Mee42/Xenon}{Github}.
    
    Xenon is a 64 bit native programming language similar to C.
    Currently written in Kotlin, a self-hosted compiler is planned for the future.
    The backend is x86\_64 NASM, a form of assembly,
    but may switch to a purely native backend eventually.
    
    The entire compiler is written by me, utalizing a regex-matching lexer, 
    a simple recursive decent parser,
    and semantic checks for type safety.
    The compiler output Intel assembly,
    and \code{nasm} and \code{gcc} is used to link in some simple standard library functions such as
    \code{malloc} and \code{printf}.


\end{document}