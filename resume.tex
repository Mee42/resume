\documentclass[10pt,letterpaper]{article}


\usepackage[letterpaper,margin=0.75in,voffset=-15pt,footskip=0pt,bottom=0cm]{geometry}
\usepackage[utf8]{inputenc}
\usepackage{mdwlist}
\usepackage[T1]{fontenc}
\usepackage{textcomp}
\usepackage{tgpagella}
\usepackage{minted}
\usepackage{multicol}
\usepackage[colorlinks=true,allcolors=.]{hyperref}
\usepackage{multicol}
\usepackage{changepage}


\pagestyle{empty}
\setlength{\tabcolsep}{0em}


\hypersetup{
 % colorlinks   = true, %Colours links instead of ugly boxes
 %lor     = blue!60!black, %Colour for external hyperlinks
 % linkcolor    = back, %Colour of internal links
 % citecolor    = red %Colour of citations
  pdftitle={Carson Graham Resume}
}


\renewcommand\labelitemi{---} % this is the list item hyphen

\def\link#1#2{\color{blue!60!black}\href{#1}{#2}\color{black}}


% make "C++" look pretty when used in text by touching up the plus signs
\newcommand{\CPP}
{C\nolinebreak[4]\hspace{-.05em}\raisebox{.35ex}{\footnotesize\bf ++}}

\newcommand{\email}
{carson42g\raisebox{-0.2ex}{@}gmail.com}

\newcommand{\github}
{github.com\hspace{-0.007em}/\hspace{-0.05em}{mee42}}

% write \code{} and have it formamtted as code. Only for inline text
\def\code#1{\texttt{#1}}

\def\paddedItem#1{\vspace{-0.4em}\item #1}

\pagestyle{myheadings}
\markright{Carson Graham\hfill}

\begin{document}
    \thispagestyle{empty}
    \begin{center}
        \huge Carson Graham
    \end{center}

    \begin{adjustwidth}{0em}{0em}
        \begin{multicols}{4}
            \begin{center}
                \link{mailto:carson42g@gmail.com}{\email}
            \end{center}
            \columnbreak
            \begin{center}
                571-577-7872
            \end{center}
            \columnbreak
            \begin{center}
                \link{https://github.com/mee42/}{\github}
            \end{center}
            \columnbreak
            \begin{center}
                Fairfax, Virginia
            \end{center}
        \end{multicols}
    \end{adjustwidth}
    \vspace{-1em}
    \begin{center}
        \code{GPG: \link{https://keys.openpgp.org/search?q=BB881A11F78A79D93FAB707D67D77A4726CF8D6F}{BB881A11F78A79D93FAB707D67D77A4726CF8D6F}}
     \end{center}

        

    \section*{Skills}

    \begin{itemize}
        \paddedItem Extensive experience with Java (5 years) and Kotlin (2 years), two JVM languages.
        \paddedItem Experience with low-level languages C, \CPP, and x86\_64 Assembly.
        \paddedItem Experience with functional programming paradigms in the forms of Haskell and Kotlin.
        \paddedItem Experience with standard GNU/Linux systems and programs, specifically Debian and Arch Linux.
        \paddedItem Experience with SQL, specifically SQLite.
        \paddedItem Experience in basic web development, HTML/CSS/JS, as well as using Kotlin/JS.
        \paddedItem Minor experience in scripting languages such as Bash and Python.
        \paddedItem Experience with build tools such as Makefile, Makepkg, and Gradle.
        \paddedItem Leadership experience (see: Robotics, CSCS)
    \end{itemize}

    \hrule
    \vspace{-0.4em}

    \section*{Leadership}
    \subsection*{Oakton HS FRC Robotics Club, Programming Lead}

    I am one of the two programming leads for the Oakton HS FRC team,
    a competitive robotics organization.
    During the season, which is 6-10 weeks, we meet about 10-30 hours a week.
    I am the main lead and, along with our 2 existing members,
    helped our 3 new programmers learn both \CPP and Git.
    Under my guidance, we adopted a mix of git-flow and trunk-based development,
    which is documented on the project \link{https://github.com/CougarProgramming623/InfiniteRecharge/wiki/Git}{wiki}.
    I was also a Dean's List Semifinalist for FIRST robotics,
    an individual competition for personal experience and accomplishments.

    \subsection*{Oakton HS Computer Science and Cyber Security Club, President}

    I am the president of the Oakton CSCS (Computer Science and Cyber Security) Club,
    one of the largest student-run clubs at my high school. 
    We have a team of 6 officers and a club of about 50 official members -- and 150 members on our public Discord server.
    We participate in Cyber Patriot, as well as several other CTFs and Computer Science competitions,
    and we have plans to host our own hackathon.
    As the president, my job is to make sure everything gets done, planning meeting agendas and club projects,
    and sharing my technical experience with other club members. 

    \vspace{1em}
    \hrule

    \section*{Formal Education and Certifications}
    \begin{itemize}
        \paddedItem AP Computer Science A, \textit{Intro to Java and Computer Science}  (Scored a 5 on the AP test freshman year)
        \paddedItem Advanced Computer Science AB, \textit{Data Structures and Algorithms}
        \paddedItem Oracle Assocate Certification, \textit{Foundations to Java SE 8} (Aquired at age 14)
        \paddedItem Oakton High School student, expected to graduate 2022 with an advanced deploma.
    \end{itemize}
    \vspace{0.3em}
    \hrule
    

    \section*{Personal Projects}

    \subsection*{Vision Pipeline, a computer vision project for pose estimation}
    \link{https://github.com/CougarProgramming623/vision-pipeline}{Github Link}. 
    
    In robotics, being able to estimate the robot's pose in relation to the target
    allows you to autonomously drive the robot better then a human can from 50 feet away.
    In the 2020 season, the goal was to shoot a ball through a hole from about 20 feet away,
    and the hope was to use a camera and some code to aim at the goal and fire with accuracy.
    This wasn't production-ready by the competition,
    but I still learned a lot about opencv and solvepnp-type problems. 

    \vspace{0.5em}
    \noindent Uses \CPP
    \hspace*{0pt}\hfill\textit{Continue on next page}

    \subsection*{Xenon, my personal programming language}
    \link{https://github.com/Mee42/Xenon}{Github Link}.
    
    Xenon is a 64 bit native programming language similar to C.
    While it is currently written in Kotlin, a self-hosted compiler is planned for the future.
    The backend is x86\_64 NASM, a form of assembly,
    but may switch to a purely native backend eventually.
    
    The entire compiler is written by me, utilizing a regex-matching lexer, 
    a simple recursive descent parser,
    and semantic checks for type safety.
    The compiler output Intel assembly,
    and \code{nasm} and \code{gcc} is used to link in some simple standard library functions such as
    \code{malloc} and \code{printf}.
    
    \vspace{0.5em}
    \noindent Uses C, Assembly, Kotlin, Xenon

    \subsection*{ADN, a content delivery server}
    
    \link{https://github.com/mee42/adn}{Github Link}.
    
    ADN is a content distribution server for quick and easy content distribution.
    I primarily designed it so I could send code and images to people through quick macros.
    This project has a server side and a CLI interface which communicates over an https API.
    The API lets users send in some text or an image
    and returns with a url that can be used to access that content again.
    When visiting the url and looking at text content - specifically, code - it will be syntax highlighted
    making ADN a great platform for sharing code.

    Sharing images can be a tedious task -- 
    this tool lets me take a screenshot with a couple keystrokes,
    then automatically puts the link on the clipboard,
    making sharing images and code painless and easy.

    \vspace{0.5em}
    \noindent Uses HTML, JS, CSS, Kotlin


\end{document}