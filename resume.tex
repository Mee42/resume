\documentclass[10pt,letterpaper]{article}


\usepackage[letterpaper,margin=0.75in]{geometry}
\usepackage[utf8]{inputenc}
\usepackage{mdwlist}
\usepackage[T1]{fontenc}
\usepackage{textcomp}
\usepackage{tgpagella}
\usepackage{minted}
\usepackage{multicol}
\usepackage{hyperref}
\usepackage{multicol}
\usepackage{changepage}


\pagestyle{empty}
\setlength{\tabcolsep}{0em}

\title{Carson Graham}
\author{}
\date{}


\hypersetup{
  colorlinks   = true, %Colours links instead of ugly boxes
  urlcolor     = blue!60!black, %Colour for external hyperlinks
  linkcolor    = blue!50!black, %Colour of internal links
  citecolor    = red %Colour of citations
}

\renewcommand\labelitemi{---} % this is the list item hyphen

\def\link#1#2{\color{blue!60!black}\href{#1}{#2}\color{black}}


% make "C++" look pretty when used in text by touching up the plus signs
\newcommand{\CPP}
{C\nolinebreak[4]\hspace{-.05em}\raisebox{.35ex}{\footnotesize\bf ++}}

\newcommand{\email}
{carson42g\raisebox{-0.2ex}{@}gmail.com}

\newcommand{\github}
{github.com\hspace{-0.007em}/\hspace{-0.05em}{mee42}}

% write \code{} and have it formamtted as code. Only for inline text
\def\code#1{\texttt{#1}}

\begin{document}

    \begin{center}
        \huge Carson Graham
    \end{center}

    \begin{adjustwidth}{0em}{0em}
        \begin{multicols}{3}
            \begin{center}
                \link{mailto:carson42g@gmail.com}{\email}
            \end{center}
            \columnbreak
            \begin{center}
                \link{https://github.com/mee42/}{\github}
            \end{center}
            \columnbreak
            \begin{center}
                Fairfax Virginia
            \end{center}
        \end{multicols}
    \end{adjustwidth}
    \vspace{-1em}
    \begin{center}
        \code{GPG: \link{https://keys.openpgp.org/search?q=BB881A11F78A79D93FAB707D67D77A4726CF8D6F}{BB881A11F78A79D93FAB707D67D77A4726CF8D6F}}
     \end{center}

        

    \section*{Skills}

    \begin{itemize}
        \item Extensive experience with Java and Kotlin, two JVM languages.
        \item Experience with low-level langauges C, \CPP, and x86\_64 Assembly.
        \item Experience with functional programming paradigms in the forms of Haskell and Kotlin.
        \item Experience with standard GNU/Linux systems and programs, specifically Debian and Arch Linux.
        \item Experience with SQL, specifically SQLite.
        \item Experience in basic web development, HTML/CSS/JS, as well as using Kotlin/JS.
        \item Minor experience in scripting languages such as Bash and Python.
        \item Experience with build tools such as Makefile, Makepkg, and Gradle.
        \item Leadership experience (see: Robotics, CSCS)
    \end{itemize}

    \hrule
    \vspace{-0.4em}

    \section*{Leadership}
    \subsection*{Oakton FRC Robotics Club}

    I am one of the two programming leads for the Oakton HS FRC team,
    a competitive robotics competition.
    During season, which is 6-10 weeks, we meet about 10-30 hours a week.
    I was the main lead and, along with our 2 existing members,
    helped our 3 new programmers learn both \CPP and Git.
    Under my guidance we adopted a mix of git-flow and trunk-based development,
    which is documented on the wiki \link{https://github.com/CougarProgramming623/InfiniteRecharge/wiki/Git}{here}.
    I was also a Dean's List Semifinalists for FIRST robotics,
    an individual competition for personal experience and accomplishments.

    \subsection*{Oakton Computer Science and Cyber Security Club}

    I'm the president of the Oakton CSCS (Computer Science and Cyber Security) Club,
    one of the largest student-run clubs at my highschool. 
    We have a team of 6 officers and a club of about 50 offical members -- and 150 members on our public Discord server.
    We go to Cyber Patriot, as well as several other CTFs and Computer Science competitions,
    and we have plans to host our own hackathon.
    As the president, my job is to make sure everything gets done, planning meeting adgendas and club projects,
    and sharing my technical experience with other club members. 

    \vspace{1em}
    \hrule

    \section*{Formal Education and Certifications}
    \begin{itemize}
        \item AP Computer Science A (Scored a 5 on the AP test freshman year), \textit{Intro to Java and Computer Science}
        \item Advanced Computer Science AB, \textit{Data Structures and Algorithms}
        \item Oracle Assocate Certification (Aquired at age 14), \textit{Foundations to Java SE 8}
    \end{itemize}

    \vspace{0.5em}
    \hrule

    \section*{Personal Projects}

    \subsection*{Vision Pipeline, a computer vision project for pose estimation}
    \href{https://github.com/CougarProgramming623/vision-pipeline}{Github}. 
    
    In robotics, being able to estimate the robot's pose in relation to the target
    allows you to autonomusly drive the robot better then a human can from 50 feet away.
    In the 2020 season, the goal was to shoot a ball through a hole from about 20 feet away,
    and the hope was to use a camera and some code to aim at the goal and fire with accuracy.
    Unfortunatly, this wasn't production-ready by the competition,
    but I still learned a lot about opencv and solvepnp-type problems. 

    \vspace{0.5em}
    \noindent Uses \CPP

    \subsection*{Xenon, my personal programming language}
    \href{https://github.com/Mee42/Xenon}{Github}.
    
    Xenon is a 64 bit native programming language similar to C.
    While it is currently written in Kotlin, a self-hosted compiler is planned for the future.
    The backend is x86\_64 NASM, a form of assembly,
    but may switch to a purely native backend eventually.
    
    The entire compiler is written by me, utalizing a regex-matching lexer, 
    a simple recursive decent parser,
    and semantic checks for type safety.
    The compiler output Intel assembly,
    and \code{nasm} and \code{gcc} is used to link in some simple standard library functions such as
    \code{malloc} and \code{printf}.
    
    \vspace{0.5em}
    \noindent Uses C, Assembly, Kotlin, Xenon

    \subsection*{ADN, a content delivery server}
    
    \href{https://github.com/mee42/adn}{Github}.
    
    \code{ADN} is a content distribution server for quick and easy content distribution.
    I primarily designed it so I couldsend code and images to people through quick macros.
    This project has a server side and a CLI interface which communicate over a https API.
    The API lets users send in some text or an image
    and returns with a url that can be used to access that content again.
    When visiting the url, text code content will by syntax highlighted,
    making it a great platform for sharing code.

    I made this because sharing images can be a tedius task -- 
    and this tool lets me take a screenshot with a couple keystrokes,
    and then automaticly puts the link it my clipboard,
    making sharing images painless and easy.

    \vspace{0.5em}
    \noindent Uses HTML, JS, CSS, Kotlin


\end{document}